% epigraph
\setlength{\unitlength}{1pt}
\setlength{\epigraphwidth}{9cm}
\epigraph{Two roads diverged in a yellow wood,\\And sorry I could not travel both...\\I took the one less traveled by,\\
And that has made all the difference.}{ --- Robert Frost, \textit{The Road Not Taken}}

% thanks to Professor (done: 2021年8月16日18:25:35)
First, I would like to thank Professor Dajun Wang for accepting me to CUHK as a postgraduate student. Without his admission, I could have already been working in the industry for five years and definitely have had no chance to see the fantastic world of physics. Dajun is a detailed-oriented professor. He can help us with any technical detail when we are freshmen in optics, electronics and mechanics. I learnt so much from him, from technique tricks to physics understanding and to the way of thinking and doing. When meeting a problem that no signal after one-day effort, I quote his word that "you cannot optimize zero." and try to change the direction to think. When seeing an unexpected signal, I always ask what he asked: "Is it a technique trick or physics?". Then I make a complete analysis instead of directly debugging instruments. After five years of training, I conclude the spirit as "always be honest to yourself".

% thanks to LG24B people (done 2021年8月16日19:34:39)
Dr. Lintao Li is the first guy I met in the lab. He is proficient in electronics and is a geek who can build any fancy circuit. Without his effort, we could still struggle in fighting with the unstable magnetic field every day. We cooperated for more than four years, and I learned a lot from him, not only electronics but also the way of debugging. He helps me conquer my fear of reading complex schematics and establish my confidence in solving any technical problem. 
Fan Jia is my close partner. We work together on almost all projects: quantum liquid droplet, Feshbach resonance calibration, detect Feshbach molecules and so on. Without his effort and company, I could not get all the things done.
I thank all people working on the mixture experiment, including Shi Yu, Tong Pan, Zerong Huang, Rongzi Zhou and Chun Kit Wong.

% thanks to LG24C people (done: 2021年8月16日19:48:56)
Another lab in our group is working on the ground state molecule of Na-Rb. Mingyang Guo and Xin Ye taught me much basic knowledge on AMO physics. Two Junyu (Junyu he and Junyu Lin) taught me knowledge of molecule and optical lattice. It is always fun to talk with Junyu. He shares lots of inspirations with me. Our conversion always tries to imagine and conceive the future of AMO physics, new technology and science and technology as an institution. All though most are only imaginary, I sincerely hope to see them within my lifetime. Besides, I want to thank other members in this lab: Mucan Jin, Guanghua Chen and two new students Zhaopeng Shi and Bo Yang. 

% Thanks to all
We work in the cold atom and molecule group as a family. I hope everyone finds their unique path and have a happy journey.

% Thanks to collaborator

% Thanks to friend of mine

% Say something about me
For my personal side, I learned from the five-year PhD career that \textit{research} is an attitude to life. 
