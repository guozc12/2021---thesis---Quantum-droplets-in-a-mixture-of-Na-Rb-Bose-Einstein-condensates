% abstract (done: 2021年8月16日10:07:06)
The Lee-Huang-Yang (LHY) correction is important for describing the Bose-Einstein condensate of dilute gases beyond the mean-field framework. In this thesis, the quantum liquid droplet, which is a direct manifestation of the LHY correction in the weakly interacting systems, is studied in the mixture of Na and Rb condensates. With the Feshbach resonance, the interspecies interaction between Na and Rb can be adjusted at will while keeping the intraspecies interaction unchanged. When the inter-species interaction is attractive enough to make the total average energy reach zero or even negative, the mixed condensate sample, instead of mean-field collapsing, stabilizes and forms a new quantum state by virtue of the LHY correction. The experiment observed the smoking-gun self-bound behaviour of the quantum droplet, i.e. maintaining a finite volume even without any confinement. The quantum liquid droplet to gaseous phase transition, which serves as a quantitative characterization of the LHY effect, is also investigated. Moreover, we explain the abnormal expansion velocity of the gaseous sample near-zero mean-field energy region as a complementary study to the liquid phase.

As the droplet phase is very sensitive to the inter-particle interactions, a precise knowledge of the scattering length is required. To this end, we performed a precise calibration of the Feshbach resonance. We measured the binding energies of NaRb Feshbach molecules by molecule dissociation. A highly accurate molecular potential energy curve is then obtained by modeling this binding energies with the coupled-channel method.  Finally, a refined mapping between the scattering length and the magnetic field is obtained. Besides the current work, this calibration also laid a solid foundation for further research with the mixture of Na and Rb atoms.