% abstract (done: 2021年8月16日10:07:06)
The mean-field theory of Bose-Einstein condensate, as a zero-order approximation, performs well under weaker interaction cases. However, when the interaction increases, we need to introduce the beyond-mean-field correction for further description, i.e. the Lee-Huang-Yang (LHY) correction. This thesis aims to experimentally study a new phase, quantum liquid droplet, formed by a two-component Bose-Einstein condensate; Correspondingly, we investigate the LHY correction, which is essential for liquid droplet formation. With the Feshbach resonance, the interspecies interaction between Na and Rb can be adjusted at will while keeping the intraspecies interaction unchanged. When the inter-species interaction is attractive enough to make the total average energy reach zero or even negative, the mixed condensate sample, instead of mean-field collapsing, stabilizes and forms a new quantum state by virtue of the LHY correction. The experiment observed a smoking-gun self-binding behaviour of the droplet state, i.e. maintaining a finite volume even without any confinement. Further experiments study the phase transition condition of the quantum liquid droplet to a gaseous phase, which serves as a quantitative characterization of the LHY effect. Moreover, we explain the abnormal expansion velocity of the gaseous sample near-zero mean-field energy region as a complementary study to the liquid phase.

Because of quantum droplets' sensitivity to the inter-particle interactions, we required a precise map of the scattering length as a function of the magnetic field. Therefore, we measured the binding energy accurately by dissociating the Na-Rb Feshbach molecules. Then, fitting by the coupled-channel method, we achieved a highly accurate molecular potential curve. Finally, we obtained a refined map of the scattering length to the magnetic field. This calibration lay a solid foundation for further researches about the mixture of Na and Rb.