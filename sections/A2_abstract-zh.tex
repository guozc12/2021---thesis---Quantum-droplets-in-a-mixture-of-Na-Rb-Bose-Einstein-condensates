% 中文摘要 (done: 2021年8月16日10:06:49)
量子多體系統的平均場理作為描述體系的零階近似在相互作用較弱的態制下表現良好。但是,隨著相互作用的增強,我們就需要引入“超越平均場”的修正來做進一步的描述。本論文旨在實驗研究由鈉和銣組成的雙組分波色-愛因斯坦凝聚形成的新的量子相——量子液滴;同時研究了解釋其所需的超越平均場的修正——也稱為“李黃楊修正”。借住費什巴赫共振,鈉銣的種間相互作用可以隨意調節,同時保持種內相互作用維持不變。當種間相互作用足夠吸引從而使得總的平均能達到零甚至是負值時,混合凝聚體系統並沒有坍縮,而是借住超越平均場的李黃楊效應穩定下來並形成了一種新的量子態——量子液滴。實驗觀察到了液滴態明確的自束縛行為,即在無外界束縛時依然保持有限體積。進一步實驗研究了量子液滴轉變為氣態的相變條件,定量的研究了李黃楊效應產生的影響。並且對於氣態樣品的反常的擴散速度給與解釋,從而揭示了更多的關於李黃楊響應對於凝聚體的影響。

借住對量子液滴的研究,我們對於表征相互作用的散射長度有了更精確的需求。所以,我們實驗通過對於費什巴赫分子的束縛能進行精確測量,從而獲得更加精確的“束縛能-磁場”關係。然後,通過耦合軌道的計算和擬合使得分子勢能曲線更加精確。從而獲得非常精確的“散射長度-磁場”關係。這一部分的研究對於未來進一步研究鈉銣混合體系的其他新奇特性打下了堅實基礎。